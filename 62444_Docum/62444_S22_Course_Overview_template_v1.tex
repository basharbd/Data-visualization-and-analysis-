%%%%%%%%%%%%%%%%%%%%%%%%%%%%%%%%%%%%%%%%%
% Beamer Presentation
% LaTeX Template
% Version 1.0 (10/11/12)
%
% This template has been downloaded from:
% http://www.LaTeXTemplates.com
%
% License:
% CC BY-NC-SA 3.0 (http://creativecommons.org/licenses/by-nc-sa/3.0/)
%
%%%%%%%%%%%%%%%%%%%%%%%%%%%%%%%%%%%%%%%%%

%----------------------------------------------------------------------------------------
%	PACKAGES AND THEMES
%----------------------------------------------------------------------------------------

\documentclass{beamer}
\mode<presentation> {

% The Beamer class comes with a number of default slide themes
% which change the colors and layouts of slides. Below this is a list
% of all the themes, uncomment each in turn to see what they look like.

%\usetheme{default}
%\usetheme{AnnArbor}  % -
%\usetheme{Antibes}   % -
%\usetheme{Bergen}    % -
%\usetheme{Berkeley}  % -
%\usetheme{Berlin}    % -
%\usetheme{Boadilla}  % -
\usetheme{boxes}  % -
%\usetheme{CambridgeUS} % -
%\usetheme{Copenhagen}
%\usetheme{Darmstadt}
%\usetheme{Dresden}
%\usetheme{Frankfurt}
%\usetheme{Goettingen}
%\usetheme{Hannover}
%\usetheme{Ilmenau}
%\usetheme{JuanLesPins}
%\usetheme{Luebeck}
%%\usetheme{Madrid}              % Option 1
%\usetheme{Malmoe}
%\usetheme{Marburg}
%\usetheme{Montpellier}
%\usetheme{PaloAlto}
%\usetheme{Pittsburgh}
%\usetheme{Rochester}
%\usetheme{Singapore}
%\usetheme{Szeged}
%\usetheme{Warsaw}

% As well as themes, the Beamer class has a number of color themes
% for any slide theme. Uncomment each of these in turn to see how it
% changes the colors of your current slide theme.

%\usecolortheme{albatross}
%\usecolortheme{beaver}
%\usecolortheme{beetle}
%\usecolortheme{crane}
\usecolortheme{dolphin}         % Option 1
%\usecolortheme{dove}
%\usecolortheme{fly}
%\usecolortheme{lily}
%\usecolortheme{orchid}
%\usecolortheme{rose}
%\usecolortheme{seagull}
%\usecolortheme{seahorse}
%\usecolortheme{whale}
%\usecolortheme{wolverine}

\setbeamertemplate{footline} % To remove the footer line in all slides uncomment this line
%\setbeamertemplate{footline}[page number] % To replace the footer line in all slides with a
                                           % simple slide count uncomment this line
%
%\setbeamertemplate{navigation symbols}{}  % To remove the navigation symbols from the bottom
                                           % of all slides uncomment this line
}

\usepackage{graphicx} % Allows including images
\usepackage{booktabs} % Allows the use of \toprule, \midrule and \bottomrule in tables
\usepackage{hyperref} 
\usepackage{romannum}
%
%----------------------------------------------------------------------------------------
%	TITLE PAGE DANISH
%----------------------------------------------------------------------------------------

%\title[Big Data]{Bigdata/Introduktion til dataanalyse og h{\aa}ndtering af store
%datam{\ae}ngder med\\ Hadoop og Spark.} % The short %title appears at the bottom of every slide,
%                                      % the full title is only on the title page
%% Big Data / Introduction to Data Analysis on Large Data Sets using Hadoop and Spark.
%%
%%
%\author{
%Jacob Nordfalk, lektor\\
%John Aa.\ S{\o}rensen, lektor}   % Your name
%%\institute[Section of Information Technology, DTU Diplom] % Your institution as it appear on bottom of every slide,
%\institute[Inf. Tech.] % Your institution as it appear on bottom of every slide,
%{
%IDA Learning \&  IDA-IT\\[5mm] % Your institution for the title page
%DTU Diplom \\ % Your institution for the title page
%Afdelingen for Informatik\\
%\medskip
%\textit{jacno@dtu.dk, jaas@dtu.dk} % Your email address
%}
%\date{8-9 Marts 2017} % Date, can be changed to a custom date
%----------------------------------------------------------------------------------------
%	TITLE PAGE ENGLISH
%----------------------------------------------------------------------------------------
\title[Course 62444]{Course 62444\\"Data Visualization and Analysis - Project"\\
using Python/Spyder/Jupyter Notebook/\\ Google CoLab,\\ R/RStudio/RStudio.Cloud \&\\ 
Julia and \LaTeX Beamer.} % The short %title appears at the bottom of every slide,
% the full title is only on the title page
% Big Data / Introduction to Data Analysis on Large Data Sets using Hadoop and Spark.
%
%
\author{
NN}   % Your name
%\institute[Section of Information Technology, DTU Diplom] % Your institution as it appear on bottom of every slide,
\institute[DTU Diplom] % Your institution as it appear on bottom of every slide,
{
%IDA Learning \&  IDA-IT\\[5mm] % Your institution for the title page
DTU Engineering Technology\\ % Your institution for the title page\\
\medskip
\textit{xyz@dtu.dk} % Your email address
}
\date{June 2022} % Date, can be changed to a custom date

%-----------------------------------------------------------------
% Insert slide numbers
\setbeamertemplate{sidebar right}{}
\setbeamertemplate{footline}{%
\hfill\usebeamertemplate***{navigation symbols}
\hspace{1cm}\insertframenumber{}/\inserttotalframenumber}
%-----------------------------------------------------------------

\begin{document}

\begin{frame}
\titlepage % Print the title page as the first slide
\end{frame}



%
%--------------------------------------------------------------------------------------------------
%
\section{Course 62444 Plan for 6 - 23 June 2022, version 1.}
\begin{frame}[fragile]
\frametitle{Course 62444 Plan for 6 - 23 June 2022, version 1. }
%\small{
%\footnotesize{
%\scriptsize{
\tiny{

\begin{tabbing}
  Thursday 6 June 8.30 - 12.00 xxxx \= Room xx  \= Subject \kill
  {\bf SCHEDULE in Room K1.03}            \>  {\bf SUBJECT} \\ \\
  Thursday 2 June, 8.00 -- 12.00 \>  {\bf Initialize individual preparation for Seminar 1}:\\
                                 \>   "Python/R and \LaTeX \hspace*{0.5mm} Tools and Platforms Review".\\
  Friday 3 June, 8.00 -- 12.00 \> Questions and  individual work with tools, references and data. \\
  Monday 6 June          \> Whitsun Holiday.\\
 Tuesday 6  June, 8.00 -- 12.00 \>  {\bf Seminar 1}: "Python/R and \LaTeX Platforms \hspace*{0.3mm} Tools Review".\\
                                 \>  After each participant has completed Seminar 1 then form groups of no\\
                                 \>   more than three persons and start preparation for Seminar 2:\\
  Tuesday 7 June, 13.00 -  \> Case Driven Visualization/Analysis using  Python/R Libraries.\\
  Wednesday 8 June, 8.00 -- 12.00 \> Questions and Group work: Visualization systems and examples.\\
  Thursday 9 June, 8.00 -- 12.00   \> Questions and Group work: Visualization systems and examples.\\
  Friday 10 June, 8.00 -- 12.00    \> Questions and Group work: Visualization systems and examples.\\
  Monday 13 June, 8.00 -- 12.00   \> Questions and Group work: Visualization system and examples.\\
  Tuesday 14 June, 8.00 -- 12.00 \> {\bf Seminar 2}: "Visualization Python/R Libraries and Examples".\\
                                 \>   After completing Seminar 2 start preparing for Seminar 3:\\
                                 
  Tuesday 14 June, 13.00--14.00  \> Case Driven Visualization/Analysis using Python/R on Single and \\
                                 \>   Mixed Category Data - Each Group Demonstrates own Examples.\\
  Wednesday 15 June, 8.00 -- 12.00 \> Questions and  Group Work: Python/R Examples on Single and $\cdots$ \\
  Thursday 16 June, 8.00 -- 12.00  \> Questions and Group Work: Python/R Examples on Single and $\cdots$ \\
  Friday 17 June, 8.00 -- 12.00    \> Questions and Group Work: Python/R Examples on Single and $\cdots$ \\
  Monday 20 June, 8.00 -- 12.00    \> Questions and Group Work: Python/R Examples on Single and $\cdots$ \\
  Tuesday 21 June, 8.00 -- 12.00 \> {\bf  Preview of Seminar 3} Final Presentation and Questions $\cdots$\\
                                              \> The project group presentation schedule is as shown on the Seminar 3 schedule.\\
  Wednesday 22 June, 8.00 -- 12.00 \> Questions and  Group Work: Python/R Examples on Single and $\cdots$ \\
  Thursday 23 June, 8.00 -- 12.00  \> {\bf Seminar 3}: "Python/R Examples on Single and Mixed Category Data\\
                                   \>  Visualization and Analysis".\\
                                   \>  Each group submits, no later than by 23.59 June 23 2022 the following:\\
                                   \>  Seminar 1, 2 and 3 presentations in pdf, the final report in pdf and\\
                                   \>  the folders 62444\_PyR and 62444\_Docum zipped.\\
                                   \>  The submission is done to the final report folder at the course CN.\\
\end{tabbing}

}
\end{frame}

%
%--------------------------------------------------------------------------------
%

\section{Seminar 1 "Python/R and \LaTeX2e \hspace*{0.5mm} Tools Review".}
\begin{frame}[fragile, label={ToolReview}]
\frametitle{Seminar 1 "Python/R and \LaTeX2e \hspace*{0.5mm} Tools Review", I}
%\small{
%\footnotesize{
%\scriptsize{
\tiny{
In preparation for Seminar 1 each individual participant should have completed the following for the R/RStudio environment:\\
Notice that this initial part is carried out individually, to ensure that all participants have verified installations of the complete selection of software packages.
\begin{itemize}
\item Installation of the R/RStudio environment following \cite{R2022}.
\item Create a shared Python/R working directory (folder) on the laptop, e.g.\ named \emph{62444\_PyR}.
\item Create a folder for documentation in \LaTeX2e, e.g.\ named \emph{62444\_Docum}, later applied for course report and presentations slides and further examples on \LaTeX2e based material.
\item Create a folder to be applied for \emph{62444\_BookShelf}.
\item Run RStudio and initialize the RStudio working directory using the following path in RStudio:
\[ \text {Session} \rightarrow \text{Set Working Directory} \rightarrow \text{Choose Directory 62444\_PyR} \]
\item Create an R script file e.g.\ named R\_1\_Example\_v1.R with the R command for getting the working directory \emph{getwd()} and
save that script in the working directory.
\item  Verify the working directory by marking that part of the script which  should be run followed by the RStudio \emph{Run} command and verifyi  that the output in the RStudio console window is  the working directory path.
\item Look for the Cheat Sheets entry at RStudio using \cite{RStudioCheatSheets2022} and download the CheatSheet \emph{RStudio IDE}. Furthermore download the \emph{Base R} cheat sheet and the \emph{\LaTeX2e} Cheat Sheet. For the latter the \emph{\LaTeX2e} code is downloaded into the folder \emph{62444\_Docum}, and it is verified that the \LaTeX2e Cheat Sheet can be compiled into a pdf using RStudio. Assign an appropriate naming to the cheat sheets and place them onto the \emph{62444\_BookShelf}. Finally download the refs.\ \cite{Torfs2014} and \cite{Oetiker2021}, rename the files according to the above library style and place them on the BookShelf.
\item Save relevant pdf references in the 62444\_BookShelf using the following file name convention:\\ 
First author family name \_ publication year\_ publication title.pdf\\
Example: VanderPlas\_2016\_A\_Whirlwind\_Tour\_of\_Python.pdf
\end{itemize}
}
\end{frame}

%-------------------------------------------------------------------------------------

\begin{frame}[fragile, label={ToolReview}]
\frametitle{Seminar 1 "Python/R and \LaTeX2e \hspace*{0.5mm} Tools Review", II}
%\small{
%\footnotesize{
\scriptsize{
%\tiny{

Furthermore in preparation for Seminar 1 each participant should have completed the following for the Python/Spyder environment:
\begin{itemize}
\item Installation of the Python/Spyder environment following \cite{Anaconda2022}.
\item Run Spyder from the Anaconda Navigator, and set the combined Python and R working directory \emph{62444\_PyR} using the Spyder meny entry in the upper rightmost corner named \emph{Browse a working directory}.
\item Create a Python script file e.g.\ named \emph{Py\_1\_Example\_v1.py} with a Python "Hello World" example and save that script in the working directory \emph{62444\_PyR} and verify in Spyder that the script works as expected.
\item Download the reference \cite{VanderPlas2016} and name the pdf file according to the FirstAuthor\_Publication Year\_Title.pdf style.
\item Study the "A Quick Tour of Python" part of \cite{VanderPlas2016} and notice that text indentation is a part of the Python code syntax. Verify code examples in the script file \emph{Py\_1\_Example\_v1.py} .
\end{itemize}

}
\end{frame}

%------------------------------------------------------------------------------------

\begin{frame}[fragile, label={ToolReview}]
\frametitle{Seminar 1 "Python/R and \LaTeX2e \hspace*{0.5mm} Tools  Review", III}
%\small{
%\footnotesize{
\scriptsize{
%\tiny{

Finally in preparation for Seminar 1 each participant should have completed the following for the \LaTeX2e \hspace*{0.2mm} environment:
\begin{itemize}
\item Each participant decides the preferred \LaTeX \hspace*{0.5mm} environment to be used (maybe Overleaf) and initiate the course report \LaTeX2e \hspace*{0.5mm} style
to be used. Furthermore the documentation and presentation of the Seminar 1 material is prepared in \LaTeX2e .
\item The \LaTeX2e \hspace*{0.5mm} Beamer Class \cite{Tantau2022} is downloaded, and named according to the above naming style and put onto the BookShelf.
\item The Seminar 1 presentation is prepared using the \LaTeX2e Beamer Class.

\end{itemize}

}
\end{frame}
%---------------------------------------------------------------------------------
\section{Seminar 1 "Presentations/Demonstrations  per Participant".}
\begin{frame}[fragile, label={ToolReview}]
\frametitle{Seminar 1 "Presentations/Demonstrations  per Participant".}
%\small{
%\footnotesize{
\scriptsize{
%\tiny{

At Seminar 1 each participant presents in a Zoom sharing the following using the \LaTeX2e \hspace*{0.5mm} Beamer environment:
\begin{itemize}
\item Verify the R/RStudio environment by exemplifying visualization of dataset e.g.\ using examples from \cite{Kabacoff2020}.    
\item Verify the Python/Spyder/Jupyter Notebook environment e.g.\ using elements from \cite{VanderPlas2016}. 
\item The \LaTeX2e \hspace*{0.5mm} Beamer Class \cite{Tantau2022} is used for presentation.
\item The folder structure applied is demonstrated, including the 62444\_BookShelf.

\end{itemize}

}
\end{frame}

%
%--------------------------------------------------------------------------------
%


\section{Seminar 1: Time Schedule.}
\begin{frame}[fragile]
\frametitle{Seminar 1: "Python/R and \LaTeX \hspace*{0.5mm} Tools Review"\\
 - Time Schedule Monday 6 June 2022, at 8.00 in room K1.03.}
%\small{
\footnotesize{
%scriptsize{
%\tiny{
Each participant presents in 5 minutes, status of their work towards completing the Tools-Review specification
as shown earlier. The presentation is followed by a short feed-back from the audience.

}
\end{frame}


%------------------------------------------------
\section{References}
\begin{frame}[allowframebreaks]
\frametitle{References}
\footnotesize{
\begin{thebibliography}{99} % Beamer does not support BibTeX so references must be inserted manually as below

\bibitem[Anaconda, 2022]{Anaconda2022} Anaconda Distribution (2022)
\newblock \emph{\href{https://www.anaconda.com}{www.anaconda.com}}
\newblock \emph{\href{https://www.anaconda.com/distribution/}{www.anaconda.com/distribution}}

\bibitem[Benoit, 2022]{Benoit2021} Kenneth Benoit (2022)
\newblock "Quantitative Analysis of Textual Data".
\newblock \emph{R-Package at CRAN 2022}
\newblock \emph{\url{https://cran.r-project.org/web/packages/quanteda/quanteda.pdf}}

\bibitem[Benoit, 2022a]{Benoit2022a} Kenneth Benoit (2022)
\newblock "vignettes quanteda: Quick Start Guide".
\emph{R-Package at CRAN 2022}
\newblock \emph{\url{https://cran.r-project.org/web/packages/quanteda/vignettes/quickstart.html}}

\bibitem[GoogleCoLab, 2022]{GoogleCoLab2022} Google CoLab (2022)
\newblock \emph{\url{https://colab.research.google.com}}

\bibitem[Horvath, 2020]{Horvath2020} Reka Horvath (2020)
\newblock "Plot With Pandas: Python Data Visualization for Beginners"
\newblock \emph{\url{https://realpython.com/pandas-plot-python/}}

\bibitem[Julia, 2022]{JuliaLang2022} The Julia Language (2022)
\newblock{\url{https://julialang.org/}}

\bibitem[Jupyter, 2021]{Jupyter2021} Jupyter Notebook (2021)
\newblock "The Jupyter Notebook".
\emph{\url{http://jupyter.org/}}

\bibitem[Kabacoff, 2020] {Kabacoff2020} Robert Kabacoff (2020)
\newblock  "Data Visualization with R".
\emph{\url{https://rkabacoff.github.io/datavis/}}

\bibitem[Machlis, 2019]{Machlis2019} Sharon Machlis (2019)
\newblock "Great R packages for data import, wrangling and visualization".
\newblock \emph{Computerworld, 2019}

\bibitem[McKinney, 2022]{McKinney2022} Wes McKinney (2022)
\newblock "pandas: powerful Python data analysis toolkit"
\newblock \emph{\url{https://pandas.pydata.org/docs/pandas.pdf }}

\bibitem[M\"{u}ller, 2020]{Muller2020} Stephan M\"{u}ller, Kenneth Banoit (2020)
\newblock "quanteda - Cheat Sheet".
\newblock \emph{\url{https://www.rstudio.com/resources/cheatsheets/}}

\bibitem[Oetiker, 2021]{Oetiker2021} Tobias Oetiker, Hubert Partl, Irene Hyna and Elisabeth Schlegl (2021)
\newblock "The Not So Short Introduction to \LaTeX2e ".
\newblock \emph{\url{https://tobi.oetiker.ch/lshort/lshort.pdf}}

\bibitem[Ognyanova 2021]{Ognyanova2021} Katherine Ognyanova (2021)
\newblock "Network visualization with R".
\newblock \emph{\url{https://kateto.net/network-visualization}}

\bibitem[Python, 2022]{PyPi2022} The Python Package Index (2022)
\newblock "The Python Package Index".
\newblock \emph{https://pypi.org/}

\bibitem[R, 2022]{R2022} The R Project for Statistical Computing (2022)
\newblock \emph{https://www.r-project.org/}

\bibitem[Ragan-Kelley, 2018]{Ragan-Kelley2018} Min Ragan-Kelley, Carol Willing, Jason Grout (2018)
\newblock "Jupyter: Tools for the Life Cycle of a Computational Idea".
\newblock \emph{SIAM News Volume 51, Issue 2 March 2018}

\bibitem[Rosling, 2018]{Rosling2018} Hans Rosling (2018)
\newblock "Gabminder".
\newblock \emph{\url{http://www.gapminder.org}}

\bibitem[RStudioCheatSheets, 2022]{RStudioCheatSheets2022} RStudio Cheat Sheets (2022)
\newblock \emph{\url{https://www.rstudio.com/resources/cheatsheets/}}

\bibitem[scikit-learn, 2022]{scikitLearn2022a} Machine Learning Map index (2022)
\newblock \emph{\url{https://scikit-learn.org/stable/tutorial/machine_learning_map/index.html}}

\bibitem[seaborn, 2022]{seaborn2022} The seaborn library for graphics in Python (2022).
\newblock \emph{\url{https://seaborn.pydata.org/}}

\bibitem[Stuart, 2022]{Stuart2022} Louis Stewart, Joseph Wright (2022)
\newblock "beamer – A LaTeX class for producing presentations and slides".
\newblock \emph{\url{https://github.com/josephwright/beamer}}

\bibitem[Tantau, 2022]{Tantau2022} Till Tantau, Joseph Wright, Vedran Mileti\'{c} (2022)
\newblock "The beamer class".
\newblock \emph{\url{http://tug.ctan.org/macros/latex/contrib/beamer/doc/beameruserguide.pdf}}


\bibitem[Torfs, 2014]{Torfs2014} Paul Torfs, Claudia Brauer (2014)
\newblock "A (very) short introduction to R".
\newblock \emph{\url{https://cran.r-project.org/doc/contrib/Torfs+Brauer-Short-R-Intro.pdf}}

\bibitem[VanderPlas, 2016]{VanderPlas2016} Jake VanderPlas (2016)
\newblock "A Whirlwind Tour of Python".
\newblock \emph{\url{https://github.com/jakevdp/WhirlwindTourOfPython}}

\end{thebibliography}
}
\end{frame}

%-----------------------------------------------------------------------------------------------------------------------------------
\end{document} 